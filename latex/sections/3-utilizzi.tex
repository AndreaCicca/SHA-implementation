\section{Utilizzi funzioni hash}
%------------------------------------------------

\begin{frame}
\frametitle{Quali utilizzi ci sono per le funzioni hash}
\begin{itemize}
    \item \textbf{Message fingerprint} (integrità dei dati)
    \item \textbf{Password hashing}
    \item \textbf{Digital signature} (solo dell'hash)
    \item \textbf{Entity Authentication/Identification} (meccanismo challenge-response)
    \item \textbf{Message Authentication} (MAC)
    \item \textbf{Encryption} (pseudorandom bit stream)
\end{itemize}
\end{frame}

\begin{frame}
\frametitle{Message Authentication Code (MAC)}

Il MAC (Message Authentication Code) è simile a una firma digitale. 
Quando si riceve un messaggio, si eseguono delle computazioni sul messaggio ricevuto e \textbf{si confronta il risultato con il MAC ricevuto}.

\begin{itemize}
    \item Conoscendo il MAC di un messaggio, è impossibile generare un altro messaggio con lo stesso MAC.
    \item È impossibile trovare due messaggi con lo stesso MAC.
    \item I MAC sono distribuiti uniformemente.
    \item Il MAC deve dipendere da ogni bit del messaggio.
\end{itemize}
\end{frame}

\begin{frame}
\frametitle{H-MAC}

MAC viene generato tramite un meccanismo di crittografia simmetrica, ma esistono anche soluzioni come H-MAC 
che sfruttano una funzione hash.

\vspace{1cm}

H-MAC esegue una funzione di hash due o tre volte a partire da una stessa chiave. In teoria, HMAC potrebbe essere 
calcolato con qualsiasi funzione hash, anche se in pratica si utilizzano quasi sempre SHA-1 o MD5 (entrambi obsoleti).


\end{frame}

\begin{frame}
    \frametitle{H-MAC Parte 2}

H-MAC viene aclcolato a blocchi e a seconda della chiave K bisognerà adattarla alla lunghezza del blocco.

\[
\text{HMAC}(K, M') = H((K' \oplus \text{opad}) \| H((K' \oplus \text{ipad}) \| M'))
\]

Dove $K'$, $\text{opad}$ e $\text{ipad}$ sono costanti.

\end{frame}